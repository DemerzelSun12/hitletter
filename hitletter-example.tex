% !TeX program = xelatex 
\documentclass[blue,harbin,pagenum,12pt]{hitletter}
\usepackage{lipsum} % 引用乱数假文宏包

% 颜色可选为blue, red, black之一
% 校区可选为harbin, shenzhen, weihai之一
% 可添加pagenum参数,添加即有页码,去掉即不显示页码

\begin{document}

HIT Letter是依照哈尔滨工业大学三个校区制作的\LaTeX 信纸模板,主要文件是hitletter.cls以及对应的校名矢量图像。制作本模板的目的是方便\TeX 用户撰写带有哈工大标志的文档/信件,免去自行设置的繁琐过程,同时尽可能符合学校的相关规定,使生成的文件更加正式、美观。

模板提供三个校区的信纸布局——harbin、shenzhen、weihai——以供用户根据不同需要选择。同时提供了页面配色功能,提供三种配色方案——blue、red、black。也考虑到部分场景需要添加页码,页码作为可选项。

模板使用方式简介:
\begin{verbatim}
	\documentclass[<COLOR>,<THEME>,<PAGENUM>,<OTHER>]{hitletter}
\end{verbatim}


模板的实现简介:

模板基于article文档类定制,使用xeCJK提供中文支持,使用graphicx+tikz+calc宏包绘制信纸部件,使用color宏包实现配色调整,使用geometry+everypage+fancyhdr宏包控制页面输出。

按照学校常用颜色,配色只设置了蓝色、红色和黑色,用户如果确实需要调整配色,可以自行修改配色方案。

注意,由于实现方式的原因,需要进行2次甚至更多次编译才能够输出最终页面。

模板的下载地址:
https://github.com/demerzelsun12/hitletter

哈尔滨工业大学校徽校名图片(hithrb.pdf 等)的版权归哈尔滨工业大学所有。

hitletter.cls 文档类与相关附属文件使用 MIT 协议授权。

~

\lipsum[1-5]

\end{document}